\documentclass[english,letter,doublesided]{article}
\usepackage{rotating}
\newcommand{\G}{\overline{C_{2k-1}}}
\usepackage[latin9]{inputenc}
\usepackage{amsmath,calligra,mathrsfs,amsfonts}
\usepackage{amssymb}
\usepackage{lmodern}
\usepackage{mathtools}
\usepackage{enumitem}
\usepackage{pgf}
\usepackage{tikz}
\usepackage{tikz-cd}
\usepackage{relsize}

\usetikzlibrary{arrows, matrix}
%\usepackage{natbib}
%\bibliographystyle{plainnat}
%\setcitestyle{authoryear,open={(},close={)}}
\let\avec=\vec
\renewcommand\vec{\mathbf}
\renewcommand{\d}[1]{\ensuremath{\operatorname{d}\!{#1}}}
\newcommand{\pydx}[2]{\frac{\partial #1}{\newcommand\partial #2}}
\newcommand{\dydx}[2]{\frac{\d #1}{\d #2}}
\newcommand{\ddx}[1]{\frac{\d{}}{\d{#1}}}
\newcommand{\hk}{\hat{K}}
\newcommand{\hl}{\hat{\lambda}}
\newcommand{\ol}{\overline{\lambda}}
\newcommand{\om}{\overline{\mu}}
\newcommand{\all}{\text{all }}
\newcommand{\valph}{\vec{\alpha}}
\newcommand{\vbet}{\vec{\beta}}
\newcommand{\vT}{\vec{T}}
\newcommand{\vN}{\vec{N}}
\newcommand{\vB}{\vec{B}}
\newcommand{\vX}{\vec{X}}
\newcommand{\vx}{\vec {x}}
\newcommand{\vn}{\vec{n}}
\newcommand{\vxs}{\vec {x}^*}
\newcommand{\vV}{\vec{V}}
\newcommand{\vTa}{\vec{T}_\alpha}
\newcommand{\vNa}{\vec{N}_\alpha}
\newcommand{\vBa}{\vec{B}_\alpha}
\newcommand{\vTb}{\vec{T}_\beta}
\newcommand{\vNb}{\vec{N}_\beta}
\newcommand{\vBb}{\vec{B}_\beta}
\newcommand{\bvT}{\bar{\vT}}
\newcommand{\ka}{\kappa_\alpha}
\newcommand{\ta}{\tau_\alpha}
\newcommand{\kb}{\kappa_\beta}
\newcommand{\tb}{\tau_\beta}
\newcommand{\hth}{\hat{\theta}}
\newcommand{\evat}[3]{\left. #1\right|_{#2}^{#3}}
\newcommand{\prompt}[1]{\begin{prompt*}
		#1
\end{prompt*}}
\newcommand{\vy}{\vec{y}}
\DeclareMathOperator{\sech}{sech}
\DeclareMathOperator{\Spec}{\mathbf{Spec}}
\DeclareMathOperator{\spec}{Spec}
\DeclareMathOperator{\spm}{Spm}
\DeclareMathOperator{\rad}{rad}
\newcommand{\mor}{\mathrm{Mor}}
\newcommand{\obj}{\mathrm{Obj}~}
\DeclarePairedDelimiter\abs{\lvert}{\rvert}%
\DeclarePairedDelimiter\norm{\lVert}{\rVert}%
\newcommand{\dis}[1]{\begin{align}
	#1
	\end{align}}
\renewcommand{\AA}{\mathbb{A}}
\newcommand{\Aa}{\mathscr{A}}
\newcommand{\LL}{\mathcal{L}}
\newcommand{\CC}{\mathbb{C}}
\newcommand{\DD}{\mathbb{D}}
\newcommand{\RR}{\mathbb{R}}
\newcommand{\NN}{\mathbb{N}}
\newcommand{\ZZ}{\mathbb{Z}}
\newcommand{\QQ}{\mathbb{Q}}
\newcommand{\Ss}{\mathcal{S}}
\newcommand{\OO}{\mathcal{O}}	
\newcommand{\BB}{\mathcal{B}}
\newcommand{\Pcal}{\mathcal{P}}
\newcommand{\FF}{\mathbb{F}}
\newcommand{\Ff}{\mathscr{F}}
\newcommand{\Gg}{\mathscr{G}}
\newcommand{\PP}{\mathbb{P}}
\newcommand{\Fcal}{\mathcal{F}}
\newcommand{\Gcal}{\mathcal{G}}
\newcommand{\fsc}{\mathscr{F}}
\newcommand{\afr}{\mathfrak{a}}
\newcommand{\bfr}{\mathfrak{b}}
\newcommand{\cfr}{\mathfrak{c}}
\newcommand{\dfr}{\mathfrak{d}}
\newcommand{\efr}{\mathfrak{e}}
\newcommand{\ffr}{\mathfrak{f}}
\newcommand{\gfr}{\mathfrak{g}}
\newcommand{\hfr}{\mathfrak{h}}
\newcommand{\ifr}{\mathfrak{i}}
\newcommand{\jfr}{\mathfrak{j}}
\newcommand{\kfr}{\mathfrak{k}}
\newcommand{\lfr}{\mathfrak{l}}
\newcommand{\mfr}{\mathfrak{m}}
\newcommand{\nfr}{\mathfrak{n}}
\newcommand{\ofr}{\mathfrak{o}}
\newcommand{\pfr}{\mathfrak{p}}
\newcommand{\qfr}{\mathfrak{q}}
\newcommand{\rfr}{\mathfrak{r}}
\newcommand{\sfr}{\mathfrak{s}}
\newcommand{\tfr}{\mathfrak{t}}
\newcommand{\ufr}{\mathfrak{u}}
\newcommand{\vfr}{\mathfrak{v}}
\newcommand{\wfr}{\mathfrak{w}}
\newcommand{\xfr}{\mathfrak{x}}
\newcommand{\yfr}{\mathfrak{y}}
\newcommand{\zfr}{\mathfrak{z}}
\newcommand{\Dcal}{\mathcal{D}}
\newcommand{\Ccal}{\mathcal{C}}
\newcommand{\Ical}{\mathcal{I}}
\newcommand{\Jcal}{\mathcal{J}}
\usepackage{graphicx}
\newcommand{\ldt}{\bullet}
\newcommand{\into}{\hookrightarrow}
% Swap the definition of \abs* and \norm*, so that \abs
% and \norm resizes the size of the brackets, and the 
% starred version does not.
%\makeatletter
%\let\oldabs\abs
%\def\abs{\@ifstar{\oldabs}{\oldabs*}}
%
%\let\oldnorm\norm
%\def\norm{\@ifstar{\oldnorm}{\oldnorm*}}
%\makeatother
\newenvironment{subproof}[1][\proofname]{%
	\renewcommand{\qedsymbol}{$\blacksquare$}%
	\begin{proof}[#1]%
	}{%
	\end{proof}%
}

\usepackage{centernot}
\usepackage{dirtytalk}
\usepackage{calc}
\newcommand{\prob}[1]{\setcounter{section}{#1-1}\section{}}


\newcommand{\prt}[1]{\setcounter{subsection}{#1-1}\subsection{}}
\newcommand{\pprt}[1]{{\textit{{#1}.)}}\newline}
\renewcommand\thesubsection{\alph{subsection}}
\usepackage[sl,bf,compact]{titlesec}
\titlelabel{\thetitle.)\quad}
\DeclarePairedDelimiter\floor{\lfloor}{\rfloor}
\makeatletter	

\newcommand*\pFqskip{8mu}
\catcode`,\active
\newcommand*\pFq{\begingroup
	\catcode`\,\active
	\def ,{\mskip\pFqskip\relax}%
	\dopFq
}
\catcode`\,12
\def\dopFq#1#2#3#4#5{%
	{}_{#1}F_{#2}\biggl(\genfrac..{0pt}{}{#3}{#4}|#5\biggr
	)%
	\endgroup
}
\def\res{\mathop{Res}\limits}
% Symbols \wedge and \vee from mathabx
% \DeclareFontFamily{U}{matha}{\hyphenchar\font45}
% \DeclareFo\newcommand{\PP}{\mathbb{P}}ntShape{U}{matha}{m}{n}{
%       <5> <6> <7> <8> <9> <10> gen * matha
%       <10.95> matha10 <12> <14.4> <17.28> <20.74> <24.88> matha12
%       }{}
% \DeclareSymbolFont{matha}{U}{matha}{m}{n}
% \DeclareMathSymbol{\wedge}         {2}{matha}{"5E}
% \DeclareMathSymbol{\vee}           {2}{matha}{"5F}
% \makeatother

%\titlelabel{(\thesubsection)}
%\titlelabel{(\thesubsection)\quad}
%\usepackage{listings}
%\lstloadlanguages{[5.2]Mathematica}
\usepackage{babel}
\newcommand{\ffac}[2]{{(#1)}^{\underline{#2}}}
\usepackage{color}
\usepackage{amsthm}
\newtheorem{thm}{Theorem}[section]
\newtheorem*{thm*}{Theorem}
\newtheorem{conj}[thm]{Conjecture}
\newtheorem{cor}[thm]{Corollary}
\newtheorem{exle}[thm]{Example}
\newtheorem{lemma}[thm]{Lemma}
\newtheorem*{lemma*}{Lemma}
\newtheorem{problem}[thm]{Problem}
\newtheorem{prop}[thm]{Proposition}
\newtheorem*{prop*}{Proposition}
\newtheorem*{cor*}{Corollary}
\newtheorem{fact}[thm]{Fact}
\newtheorem*{prompt*}{Prompt}
\newtheorem*{claim*}{Claim}
\newcommand{\claim}[1]{\begin{claim*} #1\end{claim*}}
%organizing theorem environments by style--by the way, should we really have definitions (and notations I guess) in proposition style? it makes SO much of our text italicized, which is weird.
\theoremstyle{remark}
\newtheorem{remark}{Remark}[thm]
\newtheorem*{remark*}{Remark}

\theoremstyle{definition}
\newtheorem{defn}[thm]{Definition}
\newtheorem*{defn*}{Definition}
\newtheorem{notn}[thm]{Notation}
\newtheorem*{notn*}{Notation}
%FINAL
\newcounter{hwn}
\newcounter{ryr}
\setcounter{ryr}{226}
\newcommand{\course}{8254}
\newcommand{\due}{2 (well, 8) April 2018} 
\setcounter{hwn}{5}
\RequirePackage{geometry}
\geometry{margin=.7in}
\usepackage{todonotes}
\title{MATH \course~Homework \Roman{hwn}}
\author{David DeMark}
\date{\due}
\usepackage{fancyhdr}
\pagestyle{fancy}
\fancyhf{}
\rhead{David DeMark}
\chead{\due}
\lhead{MATH \course~ Homework \Roman{hwn}}
\cfoot{\thepage}
\renewcommand{\bar}{\overline}

% %%
%%
%%
%DRAFT

%\usepackage[left=1cm,right=4.5cm,top=2cm,bottom=1.5cm,marginparwidth=4cm]{geometry}
%\usepackage{todonotes}
% \title{MATH 8669 Homework 4-DRAFT}
% \usepackage{fancyhdr}
% \pagestyle{fancy}
% \fancyhf{}
% \rhead{David DeMark}
% \lhead{MATH 8669-Homework 4-DRAFT}
% \cfoot{\thepage}

%PROBLEM SPEFICIC
\renewcommand{\hom}{\mathrm{Hom}}
\newcommand{\lint}{\underline{\int}}
\newcommand{\uint}{\overline{\int}}
\newcommand{\hfi}{\hat{f}^{-1}}
\newcommand{\tfi}{\tilde{f}^{-1}}
\newcommand{\tsi}{\tilde{f}^{-1}}

\newcommand{\nin}{\centernot\in}
\newcommand{\seq}[1]{({#1}_n)_{n\geq 1}}
\newcommand{\Tt}{\mathcal{T}}
\newcommand{\card}{\mathrm{card}}
\newcommand{\setc}[2]{\{ #1\::\:#2 \}}
\newcommand{\idl}[1]{\langle #1 \rangle}
\newcommand{\cl}{\overline}
\newcommand{\id}{\mathrm{id} }
\newcommand{\im}{\mathrm{Im}}
\newcommand{\cat}[1]{{\mathrm{\bf{#1}}}}
%\usepackage[backend=biber,style=alphabetic]{biblatex}
%\addbibresource{algeo.bib}
\newcommand{\colim}{\varinjlim}
\newcommand{\clim}{\varprojlim}
\newcommand{\frp}{\mathop{\large {\mathlarger{\star}}}}
\newcommand{\restr}[2]{{\evat{#1}{#2}{}}}
\DeclareMathOperator{\codim}{codim}
\newcommand{\imp}[1]{\underline{#1}}
\newcommand{\ihm}{\imp{\hom}}
\newcommand{\him}{\ihm(\FF,\GG)}
\newcommand{\incla}{\hookrightarrow}
\newcommand{\pre}{\mathrm{pre}}
\newcommand{\Fp}{{\FF_P}}
\renewcommand{\thethm}{\arabic{section}.\Alph{thm}}
\newcommand{\gph}{\varphi}
\newcommand{\fv}[2]{\frac{x_{#1}}{x_{#2}}}
\newcommand{\va}{\vec{a}}
\newcommand{\vai}[1]{\va^{(#1)}}
\newcommand{\csch}{\cat{Sch}}
\newcommand{\cset}{\cat{Set}}
\newcommand{\aff}{\mathrm{aff}}
%\tikzcdset{column sep/tiny=.1cm}
%\usepackage[backend=biber,style=alphabetic]{biblatex}
%\addbibresource{algeo.bib}
\DeclareMathOperator{\proj}{Proj}
\newcommand{\tpsi}{{\tilde{\psi}}}
\newcommand{\surj}{\twoheadrightarrow}
\newcommand{\tvphi}{{\tilde{\varphi}}}
\newcommand{\hol}[2]{{#1}_{[#2]}}
\newcommand{\red}{\mathrm{red}}
\newcommand{\Rfr}{\mathfrak{R}}
\newcommand{\bph}{\overline{\phi}}
\DeclareMathOperator{\lcm}{lcm}
\DeclareMathOperator{\coker}{coker}
\newcommand{\tlq}{\preccurlyeq}
\newcommand{\tlt}{\preccurly}
\newcommand{\tgq}{\succcurlyeq}
\newcommand{\tgt}{\succcurly}
\newcommand{\ny}[1]{\norm{#1}_y}
\newcommand{\ya}{y^{\avec{\alpha}}}
\newcommand{\yap}{y^{\avec{\alpha}'}}
\newcommand{\yb}{y^{\avec{\beta}}}
\newcommand{\yg}{y^{\avec{\gamma}}}
\newcommand{\blo}[2]{\mathrm{Bl}_{#1}{#2}}
\newcommand{\blzx}{\blo{Z}{X}}
\newcommand{\cech}{\v Cech~}
\newcommand{\ucal}{{ \mathcal{U}}}
\begin{document}\maketitle
	\prob{1}We let $A$ be a ring, $\afr \triangleleft A$, $X=\spec A$, $Z=\spec A/\afr$, and $\blzx=\proj A[\afr T]$.
	\begin{prop*}
		The scheme theoretic preimage $E_Z(Z):=Z\times_X \blzx$ of $Z$ w/r/t the projection $\blzx\to X$ is an effective Cartier divisor.
	\end{prop*}\begin{proof}\prt{1}
\begin{lemma}
	For any $a\in \afr$, $\Gamma(D_+(aT), \OO_{\blzx})=(A[\afr T]_{aT})_0$ is isomorphic to the subalgebra $A[a^{-1}\afr]=A\oplus a^{-1}\afr\oplus a^{-2}\afr^2\oplus \dots\subset A_a$
\end{lemma}
\begin{subproof}
We note that any element $x\in(A[\afr T]_{aT})_0$ can be written $x=\frac{bT^k}{(aT)^k}$ where $b\in\afr^k$, while any $y\in A[a^{-1}A]$ can be written $\frac{c}{a^\ell}$ where $c\in \afr^\ell$. We construct $\phi:(A[\afr T]_{aT})_0\to A[a^{-1}A]$ by $\frac{bT^k}{(aT)^k}\mapsto \frac{b}{a^k}$ and wish to show $\phi$ is an isomorphism. To show that $\phi$ is well-defined, we suppose $\frac{bT^k}{(aT)^k}=\frac{b'T^{\ell}}{(aT)^\ell}$. Then, we have that for some $r$, $$(aT)^r\left((aT)^\ell bT^k-(aT)^kb'T^\ell\right)=0$$ in $A[\afr T]$. We view $A[\afr T]$ as a subalgebra of $A[T]$. Then, we have that $$T^{\ell+k+r}\left(a^{r+\ell} b-a^{r+k}b'\right)=0$$
As $T$ is a nonzero-divisor in $A[T]$, we have that $a^{r+\ell} b-a^{r+k}b'=0$ in $A\subset A[T]$. Thus, in $A_a$, we have that $\frac{b}{a^k}=\frac{b'}{a^\ell}$, as desired, so our $\phi$ is indeed well-defined. Injectivity and surjectivity then come along easily: to see injectivity, we suppose $\phi(\frac{b}{(aT)^k})=\phi(\frac{b'}{(aT)^\ell})$, i.e. $\frac{b}{a^k}=\frac{b'}{a^\ell}$. We then have that there is some $r$ such that, in $A$, 
$$a^r(ba^{\ell}-b'a^k)=0$$ Then, in $A[T]$, we multiply through by $T^{k+\ell+r}$ to yield
$$T^{k+\ell+r}a^r(ba^{\ell}-b'a^k)=(aT)^r\left((aT)^\ell bT^k-(aT)^kb'T^\ell\right)=0$$
Thus, we have that $\frac{bT^k}{(aT)^k}=\frac{b'T^{\ell}}{(aT)^\ell}$. Finally, surjectivity follows obviously from what we have done so far. If $x=\frac{bT}{a^k}\in A[a^{-1}\afr],$ then, $x=\phi(\frac{bT^k}{(aT)^k})$.
\end{subproof}
\end{proof}
	\prob{4}\emph{I'm only convinced that this is even true if }
	\prob{5}
\prt{1} \begin{prompt*}
Determine the \cech cohomology of the structure sheaf of $X$, the affine line with doubled origin with respect to the usual open affine cover $\ucal$ consisting of two open affine subsets isomorphic to $\AA^1_k$.
\end{prompt*}
\begin{proof}[Computation]
We name our two open affines $U_0$ and $U_1$ with each $U_i\cong \spec k[x]=\AA^1_k$. We have that $C^0(\ucal,\OO_X)=\OO_X(U_1)\otimes \OO_X(U_2)=k[x]\oplus k[x]$. We also have only one nontrivial twofold intersection of the sets in our cover and thus an element of $C^1(\ucal,\OO_X)$ is uniquely determined by its $U_{01}$ component, that is $C^1(\ucal,\OO_X)=\OO_X(U_{01})=k[x,x^{-1}]$. We now compute our only nontrivial differential $d^0$ on an arbitrary element $(g_0(x),g_1(x))\in C^0(\ucal,\OO_X)$ and have that $d^0:(g_0(x),g_1(x))=g_1(x)-g_0(x)$ as our restriction maps are both the standard localization maps $x\mapsto x$. Thus, $H^0(\ucal,\OO_X)=\ker d^0$, which is the diagonal subring $D(k[x]\oplus k[x])=\{(g(x),g(x)): g(x)\in k[x]\}\cong k[x]$. We also have that (as $C^2=0$) $H^1(\ucal, \OO_X)=\coker d^0=k[x,x^{-1}]/k[x]$ as an Abelian group (or module over the ring of global sections $k[x]$). \end{proof}
\begin{prompt*}
	Determine the \cech cohomology of the structure sheaf of $\PP^1_k$ with respect to the usual open affine cover $\ucal$ consisting of two open affine subsets $D_+(x_0),D_+(x_1)$ isomorphic to $\AA^1_k$.
\end{prompt*}
\begin{proof}[Computation]
	We keep our setup from before, only changing our restriction map $\rho_{01}^1$ on $U_1$ to the map $x\mapsto x^{-1}$. We have still that $C^0(\ucal,\OO_X)=\OO_X(U_1)\otimes \OO_X(U_2)=k[x]\oplus k[x]$ and $C^1(\ucal,\OO_X)=\OO_X(U_{01})=k[x,x^{-1}]$. We now compute our only nontrivial differential $d^0$ on an arbitrary element $(g_0(x),g_1(x))\in C^0(\ucal,\OO_X)$ and have that $d^0:(g_0(x),g_1(x))=g_1(x^{-1})-g_0(x)$, with the $x^{-1}$ in the argument of $g_1$ forced by our restriction map $\rho_{01}^1$. We note that if $g_1$ is homogenous of positive degree, then $g_1(x^{-1})$ is homogenous of negative degree, while $g_0(x)$ is of nonnegative degree. Thus, only degree 0 elements are in the kernel of $d^0$, that is $\ker d^0=H^0(\ucal,\OO_X)=k$. We claim that $d^0$ is a surjection. Indeed, $C_1(\ucal,\OO_X)$ is the ring of Laurent polynomials in $k$; we let $g(x)=a_mx^m+\dots+a_{-n}x^{-n}$ be an arbitrary element. Then, we let $g_1(x)=\sum_{k=0}^na_{-k}x^k$ and $g_0(x)=\sum_{k=1}^m-a_{k}x^k$ and have that $d^0(g_0,g_1)=g$, thus showing that $H^1(\ucal, \OO_X)=0$. \end{proof}
\end{document}